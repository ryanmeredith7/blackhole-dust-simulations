\documentclass[12pt]{article}
\usepackage{parskip}
\frenchspacing

\usepackage{hyperref}
\usepackage{fontspec}
\setmainfont{EB Garamond}

\usepackage[intlimits]{mathtools}
\usepackage{amssymb}
\newcommand{\dd}{\mathop{}\!\mathrm{d}}

\title{Solving the PDEs}
\author{Ryan Meredith}

\begin{document}

\maketitle

\section{Introduction}

Starting from the equations
\[ \dot{b} = \frac{\gamma x}{2(E^b)^2} - \frac{\gamma}{2x} - \frac{x}{\gamma\Delta} \sin\left(
\frac{\sqrt{\Delta}}{x}b \right) \left( \frac{3}{2}\sin\left( \frac{\sqrt{\Delta}}{x}b \right) +
\partial_x \left( \sin\left(\frac{\sqrt{\Delta}}{x}b\right) \right) \right) \]
and
\[ \dot{E^b} = -\frac{x^2}{\gamma\sqrt{\Delta}} \partial_x \left( \frac{E^b}{x} \right) \sin \left(
\frac{\sqrt{\Delta}}{x}b \right) \cos \left( \frac{\sqrt{\Delta}}{x}b \right) \]
we can introduce the variables
\[ \beta = \frac{\sqrt{\Delta}}{x}b \]
and
\[ \alpha = \gamma\sqrt{\Delta} \left( \frac{1}{x^2} - \frac{1}{(E^b)^2} \right). \]
so
\[ \dot{\alpha} = -\frac{x\sin(2\beta)}{2\gamma\sqrt{\Delta}}\alpha_x -
\frac{\sin(2\beta)\alpha}{\gamma\sqrt{\Delta}} \]
and
\[ \dot{\beta} = -\frac{x\sin(2\beta)}{2\gamma\sqrt{\Delta}}\beta_x -
\frac{3\sin^2(\beta)}{2\gamma\sqrt{\Delta}} - \frac{\alpha}{2} \]

\section{Numerics}

From here on we will set $\Delta=\gamma=1$, thus we have the equations
\[ \dot{\alpha} = -\frac{x}{2}\sin(2\beta)\alpha_x - \sin(2\beta)\alpha \]
and
\[ \dot{\beta} = -\frac{x}{2}\sin(2\beta)\beta_x - \frac{3}{2}\sin^2(\beta) - \frac{\alpha}{2} \]
To solve this numerically, we must first break the domain up into cells. Let $x_0$ be the start of
the domain and $x_n = x_0 + n\Delta x$, then we have cells $C_n = [x_n,x_{n+1}]$. Next, at the
initial time $t_1$, we approximate $\alpha$ and $\beta$ as constant inside each cell, that is for
$x\in(x_n,x_{n+1})$ we have $\alpha(x,t_1)=\alpha_n^1$ and $\beta(x,t_1) = \beta_n^1$.

\subsection{Wave Speeds}

Following Randall J. Leveque's \textit{Finite-Volume Methods for Hyperbolic Problems}, we can view
the discontinuities between cells as waves and then calculate the wave speeds to propagate them into
the neighboring cells to calculate the solution after a time step $\Delta t$. As long as the wave is
moving entirely in one direction, whether it is a shock wave or a rarefaction wave, we can simply
set the wave speeds to
\[ s_n = \begin{dcases}
    \frac{x}{2} \frac{\sin^2(\beta_n^1) - \sin^2(\beta_{n-1}^1)}{\beta_n^1 - \beta_{n-1}^1},
    & \beta_{n-1}^1 \ne \beta_n^1 \\
    \frac{x}{2} \sin(2\beta_n^1), & \beta_{n-1}^1 = \beta_n^1
\end{dcases} \]
but in the case of a transonic rarefaction wave, where part of the wave moves to the left and part
moves to the right, things get a little more complicated.

To aid with this, we can simply assume there is up to two waves at each cell interface, a left
moving one and a right moving one. For the case where there is only a wave moving in one direction,
we can simply set the wave speed in the other direction to 0. This could be done like this
\[ s^l_n = \min(s_n,0) \]
\[ s^r_n = \max(s_n,0) \]
but it's probably easier computationally to simply partition them based on sign into new arrays
whose default value is 0. The transonic rarefaction waves happen in two cases, if $\beta_{n-1}^1 <
\beta_n^1$ and the speed switches from negative to positive, or if $\beta_{n-1}^1 > \beta_n^1$ and
the speed switches from positive to negative. Since $x$ is always positive, we need only worry about
when $\sin(2\beta)$ changes sign, which we can use quadrants for.

The level sets of the function $\left\lfloor \frac{x}{\pi} \right\rfloor$ are cycles of the function
$\sin(2x)$, and the transition between cycles is the only place $\sin$ changes sign from negative to
positive, so the first type of transonic rarefaction wave happens if and only if
\[ \left\lfloor \frac{\beta_{n-1}^1}{\pi} \right\rfloor < \left\lfloor \frac{\beta_n^1}{\pi}
\right\rfloor \]
Now the speed of the leftward moving wave can be calculated by taking the average speed of the
portion of the wave before the first change from negative to positive, and the speed of the
rightward moving wave can be calculated by taking the average speed of the portion of the wave after
the last change from negative to positive. This takes the form
\[ s^l_n = \frac{x_n}{2} \frac{\sin^2(\beta^*) - \sin^2(\beta_{n-1}^1)}{\beta^* - \beta_{n-1}^1} \]
\[ s^r_n = \frac{x_n}{2} \frac{\sin^2(\beta_n^1) - \sin^2(\beta^*)}{\beta_n^1 - \beta^*} \]
where $\beta^*$ is the points points where the speed changes from negative to positive, which may or
may not be the same in the two equations. However, we do not actually need to calculate $\beta^*$
since we know $\sin^2(\beta^*)=0$ and we only need to get the proper fluctuations from the
propagating waves. To do this we can simply scale the wave speed so that it will act like a wave of
the full magnitude, and then use the same method as for the full magnitude waves we calculated
before. To do this we can simply use
\[ s^l_n = \frac{x_n}{2} \frac{\sin^2(\beta_{n-1}^1)}{\beta_{n-1}^1 - \beta_n^1} \]
\[ s^r_n = \frac{x_n}{2} \frac{\sin^2(\beta_n^1)}{\beta_n^1 - \beta_{n-1}^1} \]

The same arguments apply for the other type of transonic rarefaction waves, which will happen if and
only if
\[ \left\lfloor \frac{\beta_{n-1}^1}{\pi} + \frac{1}{2} \right\rfloor > \left\lfloor
\frac{\beta_n^1}{\pi} + \frac{1}{2} \right\rfloor \]
and we can use the speeds
\[ s^l_n = \frac{x_n}{2} \frac{1 - \sin^2(\beta_{n-1}^1)}{\beta_n^1 - \beta_{n-1}^1} \]
\[ s^r_n = \frac{x_n}{2} \frac{1 - \sin^2(\beta_n^1)}{\beta_{n-1}^1 - \beta_n^1} \]

So for speeds we have
\[ s^l_n = \begin{dcases}
    \frac{x_n}{2} \frac{\sin^2(\beta_{n-1}^1)}{\beta_{n-1}^1 - \beta_n^1}, & \left\lfloor
    \frac{\beta_{n-1}^1}{\pi} \right\rfloor < \left\lfloor \frac{\beta_n^1}{\pi} \right\rfloor \\
    \frac{x_n}{2} \frac{1 - \sin^2(\beta_{n-1}^1)}{\beta_n^1 - \beta_{n-1}^1}, & \left\lfloor
    \frac{\beta_{n-1}^1}{\pi} + \frac{1}{2} \right\rfloor > \left\lfloor \frac{\beta_n^1}{\pi} +
    \frac{1}{2} \right\rfloor \\
    \min(s_n, 0), & \text{otherwise}
\end{dcases} \]
and
\[ s^r_n = \begin{dcases}
    \frac{x_n}{2} \frac{\sin^2(\beta_n^1)}{\beta_n^1 - \beta_{n-1}^1}, & \left\lfloor
    \frac{\beta_{n-1}^1}{\pi} \right\rfloor < \left\lfloor \frac{\beta_n^1}{\pi} \right\rfloor \\
    \frac{x_n}{2} \frac{1 - \sin^2(\beta_n^1)}{\beta_{n-1}^1 - \beta_n^1}, & \left\lfloor
    \frac{\beta_{n-1}^1}{\pi} + \frac{1}{2} \right\rfloor > \left\lfloor \frac{\beta_n^1}{\pi} +
    \frac{1}{2} \right\rfloor \\
    \max(s_n, 0), & \text{otherwise}
\end{dcases} \]

A note about these formula before we move on, the structure of the wave can be quite complicated,
possibly involving multiple shock and rarefaction waves mix and matched together, but here we simply
take the average speed of the waves moving to the left and right, so it will work no matter the
structure of the wave.

\subsection{Update Formula}

The fluctuations from the propagating waves after a time $\Delta t$ can be calculated by
\[ \frac{\Delta t}{\Delta x} (s^r_n (\beta_n^1 - \beta_{n-1}^1) + s^l_{n+1} (\beta_{n+1}^1 -
\beta_n^1)) \]
for $\beta$, and
\[ \frac{\Delta t}{\Delta x} (s^r_n (\alpha_n^1 - \alpha_{n-1}^1) + s^l_{n+1} (\alpha_{n+1}^1 -
\alpha_n^1)) \]
for $\alpha$. For further details on this, see Leveque's book. The other terms in the equation are
source terms, and we will do the simplest method to deal with them, just calculate the term for each
cell and multiply by the time step. Letting $\alpha_n^2 = \alpha(x,t_1 + \Delta t)$ and $\beta_n^2 =
\beta(x,t_1 + \Delta t)$ for $x \in C_n$, we have
\[ \alpha_n^2 = \alpha_n^1 - \Delta t \left( \frac{s^r_n (\alpha_n^1 - \alpha_{n-1}^1) + s^l_{n+1}
(\alpha_{n+1}^1 - \alpha_n^1)}{\Delta x} + \sin(2\beta_n^1)\alpha_n^1 \right) \]
and
\[ \beta_n^2 = \beta_n^1 - \Delta t \left( \frac{s^r_n (\beta_n^1 - \beta_{n-1}^1) + s^l_{n+1}
(\beta_{n+1}^1 - \beta_n^1)}{\Delta x} + \frac{3}{2}\sin^2(\beta_n^1) + \frac{\alpha_n}{2} \right)
\]
You could also potentially use a fractional step method to deal with the source terms, but I did not
test it. Another potential improvement would be to add higher order correction terms. Both of these
are outlined and discussed in Leveque's book.

As a final note, here we assumed there is a linear relationship between the waves in $\alpha$ and
$\beta$, for example, if there is a transonic rarefaction wave where half of $\beta$ move left and
the other half moves right, then we assume that half of $\alpha$ move left and the other half moves
right. We did this for simplicity and because it appears to work, but it could be beneficial to test
other relationship.

\end{document}
