\documentclass[12pt]{beamer}
\usepackage{multimedia}
\usefonttheme{serif}

\usetheme{Madrid}
\beamertemplatenavigationsymbolsempty

\usepackage[intlimits]{mathtools}
\usepackage{amssymb}
\newcommand{\dd}{\mathop{}\!\mathrm{d}}

\title[Simulating Black Holes]{Simulating Black Holes as Dust Clouds with Loop Quantum
Gravity Correction}
\author{Ryan Meredith}

\begin{document}

\frame{\titlepage}

\begin{frame}{Introduction}
    This project consists of:
    \begin{itemize}
        \item Lema\^{i}tre-Tolman-Bondi metric in Painlev\'{e}-Gullstrand coordinates
            with loop quantum gravity corrections
        \item Numeric solver of equations of motion
        \item Animations of several initial conditions
        \item Timer of event horizon lifetimes
    \end{itemize}
\end{frame}

\begin{frame}{Background}
    \begin{itemize}
        \item A black hole is a region of space-time which is separated from infinity by
            an event horizon
        \item An event horizon is a surface across which there are no time-like paths
            from the interior to infinity
        \item In essence, an event horizon is a surface particles can enter but not
            escape
        \item Classically, there is a singularity at the center of the black hole, where
            the curvature is infinite
        \item With loop quantum gravity, there is a maximum density imposed, so there is
            no longer a singularity at the center of the black hole
    \end{itemize}
\end{frame}

\begin{frame}{Background}
    Lema\^{i}tre-Tolman-Bondi metric:
    \[ \dd s^2 = -\dd t^2 + \frac{\dot{R}(t,r)^2}{1 + 2E(r)} \dd r^2 + R^2\dd \Omega^2 \]
    In Painlev\'{e}-Gullstrand coordinates:
    \[ \dd s^2 = -\dd t^2 + \frac{\left(\dd R^2 + \sqrt{E(t,R) + \frac{F(t,R)}{R}}\dd
    t\right)^2} {1 + E(t,R)} + R^2\dd \Omega^2 \]
    With loop quantum gravity corrections:
    \[ \dd s^2 = -\dd t^2 + \frac{E^b(x,t)^2}{x^2} (\dd x + N^x\dd t)^2 + x^2 \dd
    \Omega^2 \]
    \[ N^x = -\frac{x}{\gamma\sqrt{\Delta}}
    \sin\left(\frac{\sqrt{\Delta}b(x,t)}{x}\right)
    \cos\left(\frac{\sqrt{\Delta}b(x,t)}{x}\right) \]
\end{frame}

\begin{frame}{Equations of motion}
    \begin{multline*}
        \dot{b} = \frac{\gamma x}{2(E^b)^2} - \frac{\gamma}{2x} - \frac{x}{\gamma\Delta}
        \sin\left( \frac{\sqrt{\Delta}}{x}b \right) \\
        \cdot \left( \frac{3}{2}\sin\left( \frac{\sqrt{\Delta}}{x}b \right) + \partial_x
        \left( \sin\left(\frac{\sqrt{\Delta}}{x}b\right) \right) \right)
    \end{multline*}
    \[ \dot{E^b} = -\frac{x^2}{\gamma\sqrt{\Delta}} \partial_x \left( \frac{E^b}{x}
    \right) \sin \left( \frac{\sqrt{\Delta}}{x}b \right) \cos \left(
    \frac{\sqrt{\Delta}}{x}b \right) \]
\end{frame}

\begin{frame}{Change of Variables}
    For the numerics, let $\gamma=\Delta=1$.

    \[ \alpha = \frac{1}{x^2} - \frac{1}{(E^b)^2} \]
    \[ \beta = \frac{b}{x} \]

    \[ \dot{\alpha} = -\frac{x}{2}\sin(2\beta)\alpha_x - \sin(2\beta)\alpha \]
    \[ \dot{\beta} = -\frac{x}{2}\sin(2\beta)\beta_x - \frac{3}{2}\sin^2(\beta) -
    \frac{\alpha}{2} \]
\end{frame}

\begin{frame}{PDE Solver}
    \begin{itemize}
        \item Based on \textit{Finite Volume Methods for Hyperbolic Problems} by Randall
            J.\ LeVeque
        \item Solutions emit shock waves, so a modified Godunov's method was used
        \item The solution domain was split up into cells
        \item Initial function approximated as piecewise constant functions
        \item Waves emanate from each cell interface
        \item Wave speed given by
            \[ \frac{x}{2}\sin(2\beta) \]
        \item Does not depend on $\alpha$
    \end{itemize}
\end{frame}

\begin{frame}{Wave Speeds}
    \begin{itemize}
        \item Wave structure can be complicated
        \item Create an average wave to each direction at each cell interface
        \item Normalize the wave speeds to always use $\beta_n - \beta_{n-1}$ and
            $\alpha_n - \alpha_{n-1}$ as the size of the waves for $\beta$ and $\alpha$
            respectively
        \item At the interface of cell $n-1$ and $n$
            \[ s_{n-1/2} = \begin{dcases}
                \frac{x_{n-1/2}}{2} \frac{\sin^2(\beta_n) - \sin^2(\beta_{n-1})}{\beta_n
                - \beta_{n-1}}, & \beta_n \ne \beta_{n-1} \\
                \frac{x_{n-1/2}}{2} \sin(2\beta_n), & \beta_n = \beta_n{-1}
            \end{dcases} \]
    \end{itemize}
\end{frame}

\begin{frame}{Wave Speeds}
    \[ s_{n-1/2}^l = \begin{dcases}
        \frac{x_{n-1/2}}{2} \frac{\sin^2(\beta_{n-1})}{\beta_{n-1} - \beta_n}, &
        \left\lfloor \frac{\beta_{n-1}}{\pi} \right\rfloor < \left\lfloor
        \frac{\beta_n}{\pi} \right\rfloor \\
        \frac{x_{n-1/2}}{2} \frac{\cos^2(\beta_{n-1})}{\beta_n - \beta_{n-1}}, &
        \left\lfloor \frac{\beta_{n-1}}{\pi} - \frac{1}{2} \right\rfloor  > \left\lfloor
        \frac{\beta_n}{\pi} - \frac{1}{2} \right\rfloor \\
        \min(s_{n-1/2}, 0), & \textrm{otherwise}
    \end{dcases} \]
    \[ s_{n-1/2}^r = \begin{dcases}
        \frac{x_{n-1/2}}{2} \frac{\sin^2(\beta_n)}{\beta_n - \beta_{n-1}}, &
        \left\lfloor \frac{\beta_{n-1}}{\pi} \right\rfloor < \left\lfloor
        \frac{\beta_n}{\pi} \right\rfloor \\
        \frac{x_{n-1/2}}{2} \frac{\cos^2(\beta_n)}{\beta_{n-1} - \beta_n}, &
        \left\lfloor \frac{\beta_{n-1}}{\pi} - \frac{1}{2} \right\rfloor  > \left\lfloor
        \frac{\beta_n}{\pi} - \frac{1}{2} \right\rfloor \\
        \max(s_{n-1/2}, 0), & \textrm{otherwise}
    \end{dcases} \]
\end{frame}

\begin{frame}{Update Formulas}
    \begin{multline*}
        \alpha_n^{new} = \alpha_n - \Delta t \Biggl( \frac{s_{n-1/2}^r (\alpha_n -
        \alpha_{n-1}) + s_{n+1/2}^l (\alpha_{n+1} - \alpha_n)}{\Delta x} \\
        + \sin(2\beta_n)\alpha_n \Biggr)
    \end{multline*}
    \begin{multline*}
        \beta_n^{new} = \beta_n - \Delta t \Biggl( \frac{s_{n-1/2}^r (\beta_n -
        \beta_{n-1}) + s_{n+1/2}^l (\beta_{n+1} - \beta_n)}{\Delta x} \\
        + \frac{3}{2}\sin^2(\beta_n) + \frac{\alpha_n}{2} \Biggr)
    \end{multline*}
\end{frame}

\begin{frame}{Boundary Condition}
    Inner boundary irrelevant, wave speed is zero
    \[ \beta_0 = \beta_1 \]
    \[ \alpha_0 = \alpha_1 \]

    Outer boundary, $\dot{\alpha} = \dot{\beta} = 0$
    \[ \beta_{N+1} = -\sin^{-1}\left( \sqrt{\sin^2(\beta_N) - \frac{1}{N}
    (3\sin^2(\beta_N) + \alpha_N)} \right) \]
    \[ \alpha_{N+1} = \frac{1 + 2 (\beta_{N+1} - \beta_N)
    \sin(2\beta_N)}{3\sin^2(\beta_N) + \alpha_N} \alpha_N \]
\end{frame}

\begin{frame}{Dynamics and Animations}
    \begin{itemize}
        \item I will be showing the animations generated by my program
        \item I will explain our expectations
        \item I will point out anything interesting we saw in the animations
    \end{itemize}
\end{frame}

\begin{frame}[shrink]{No Curvature $\alpha$}
    \centering
    \movie{\includegraphics[width=0.9\textwidth]{../plots/flatA.png}}{../plots/flatA.avi}
\end{frame}
\begin{frame}[shrink]{No Curvature $\beta$}
    \centering
    \movie{\includegraphics[width=0.9\textwidth]{../plots/flatB.png}}{../plots/flatB.avi}
\end{frame}

\begin{frame}[shrink]{No Curvature $\rho$}
    \centering
    \movie{\includegraphics[width=0.9\textwidth]{../plots/flatP.png}}{../plots/flatP.avi}
\end{frame}

\begin{frame}[shrink]{Simple Curvature $\alpha$}
    \centering
    \movie{\includegraphics[width=0.9\textwidth]{../plots/testA.png}}{../plots/testA.avi}
\end{frame}
\begin{frame}[shrink]{Simple Curvature $\beta$}
    \centering
    \movie{\includegraphics[width=0.9\textwidth]{../plots/testB.png}}{../plots/testB.avi}
\end{frame}

\begin{frame}[shrink]{Simple Curvature $\rho$}
    \centering
    \movie{\includegraphics[width=0.9\textwidth]{../plots/testP.png}}{../plots/testP.avi}
\end{frame}

\begin{frame}[shrink]{Sinusoidal Curvature $\alpha$}
    \centering
    \movie{\includegraphics[width=0.9\textwidth]{../plots/test2A.png}}{../plots/test2A.avi}
\end{frame}
\begin{frame}[shrink]{Sinusoidal Curvature $\beta$}
    \centering
    \movie{\includegraphics[width=0.9\textwidth]{../plots/test2B.png}}{../plots/test2B.avi}
\end{frame}

\begin{frame}[shrink]{Sinusoidal Curvature $\rho$}
    \centering
    \movie{\includegraphics[width=0.9\textwidth]{../plots/test2P.png}}{../plots/test2P.avi}
\end{frame}

\begin{frame}[shrink]{Sawtooth Curvature $\alpha$}
    \centering
    \movie{\includegraphics[width=0.9\textwidth]{../plots/test3A.png}}{../plots/test3A.avi}
\end{frame}

\begin{frame}[shrink]{Sawtooth Curvature $\beta$}
    \centering
    \movie{\includegraphics[width=0.9\textwidth]{../plots/test3B.png}}{../plots/test3B.avi}
\end{frame}

\begin{frame}[shrink]{Sawtooth Curvature $\rho$}
    \centering
    \movie{\includegraphics[width=0.9\textwidth]{../plots/test3P.png}}{../plots/test3P.avi}
\end{frame}

\begin{frame}[shrink]{Random Curvature $\alpha$}
    \centering
    \movie{\includegraphics[width=0.9\textwidth]{../plots/testRA.png}}{../plots/testRA.avi}
\end{frame}

\begin{frame}[shrink]{Random Curvature $\beta$}
    \centering
    \movie{\includegraphics[width=0.9\textwidth]{../plots/testRB.png}}{../plots/testRB.avi}
\end{frame}

\begin{frame}[shrink]{Random Curvature $\rho$}
    \centering
    \movie{\includegraphics[width=0.9\textwidth]{../plots/testRP.png}}{../plots/testRP.avi}
\end{frame}

\begin{frame}{Event Horizon Lifetimes}
    \begin{itemize}
        \item Event horizon exists where $\theta$ is zero
            \[ \theta = 1 - x^2 \left( \alpha + \frac{\sin^2(2\beta)}{4} \right) \]
        \item Created a function which times the lifetime of event horizons
        \item Tested on the flat special case
        \item Expected event horizon lifetimes to depend quadratically on the mass of
            the black hole, with leading coefficient $\frac{8}{3}\pi = 8.37758$
    \end{itemize}
\end{frame}

\begin{frame}{Fitted Data}
    \centering
    \includegraphics[width=0.7\textwidth]{../plots/fit.png}
    \[ t = 2.307m^2 + 29.299m - 63.057 \]
    \[ R^2 = 0.99327 \]
\end{frame}

\begin{frame}{The Issue}
    \begin{itemize}
        \item The speed of shock waves is highly dependant on the exact form of the
            equations being solved
        \item Mathematically equivalent forms of equations will give different solutions
            numerically
        \item Especially important is choice of variables
        \item I used a non-conservative form of the equations because they were easier
            to implement
        \item Speed of shock waves were too fast
    \end{itemize}
\end{frame}

\begin{frame}{Next Steps}
    \begin{itemize}
        \item Rewrite program to use variable $B = xb = x^2\beta$, this will give a
            conservation law with a source term for one of the equations
        \item Fix boundary condition issue
        \item Look at event horizon lifetimes for curved cases
        \item Examine more dynamics of curved cases
        \item See if higher order correction term improves results
        \item See if using a fractional step method improves results
    \end{itemize}
\end{frame}

\begin{frame}{Conclusion}
    \begin{itemize}
        \item Got equations of a black hole as a dust cloud and added loop quantum
            gravity corrections
        \item Created a numeric solver for the equations and used it in a simulation
            program
        \item Found some issues with the solver
        \item Still showed interesting dynamics, which we believe are accurate, just on
            a incorrect time scale
    \end{itemize}
\end{frame}

\end{document}
