\documentclass[12pt]{article}
\usepackage{parskip}
\usepackage{fontspec}
\setmainfont{EB Garamond}
\usepackage{hyperref}
\frenchspacing

\usepackage[intlimits]{mathtools}
\usepackage{amssymb}
\newcommand{\dd}{\mathop{}\!\mathrm{d}}

\usepackage{biblatex}
\addbibresource{sources.bib}

\title{Simulating Black Holes as Dust Clouds with Loop Quantum Gravity Corrections}
\author{Ryan Meredith\\3559441}

\begin{document}

\maketitle

\tableofcontents

\section{Introduction}

In this project, a solver for the equations of motion of a spherically symmetric dust
cloud, with loop quantum gravity corrections applied, was created to simulate black
holes. Then, that solver was applied to a variety of initial conditions to examine the
dynamics they resulted in. Finally, the solver was used to time the lifetimes of event
horizons and examine them as a function of mass. The source code of this project is
available at \url{https://github.com/ryanmeredith7/blackhole-dust-simulations}.

The foundations of this project are in general relativity. The metric of interest is the
Lema\^{i}tre-Tolman-Bondi metric in Painlev\'{e}-Gullstrand coordinates. The equations of
motion are extracted from this setting, but loop quantum gravity corrections are added
to the equations. Unfortunately, the details of loop quantum gravity are beyond the
scope of this project.

The solver is necessarily complicated since the equation of motions are a non-linear set
of coupled partial differential equations. The solver uses a modified version of
Godunov's method for conservation laws. It also uses the viewpoint of wave speeds and
fluctuations so that the wave speeds can be calculated once and then applied to both
equations, which simplifies the calculations a lot since the wave speeds only depend on
one of the variables.

Once the solver is working, we can apply a variety of initial conditions to test various
interesting cases. Using MATLAB's visualization tools, we can create animations of the
solutions to examine the dynamics of the solution qualitatively. We can also compare the
behavior of the solutions to what we expect based on the differential equations
themselves.

Finally, we can also examine the solutions quantitatively by using the solver to time
the lifetimes of the event horizons of the black holes. Previous work has been done on a
special case of these equations of motion, so we can compare to that to validate the
solver. The idea would be to then extend the model from the special case to the general
case, but there was unfortunately not enough time to do that.

\section{Background Material}

The foundation of my project is general relativity. There are many great textbooks on
general relativity that could be used as reference, I used Sean Carroll's
\textit{Spacetime and Geometry: An Introduction to General Relativity} \cite{gr}.

The metric used for this project is the Lema\^{i}tre-Tolman-Bondi metric. This metric is a
spherically symmetric dust cloud solution to Einstein's field equations. The metric in
Lema\^{i}tre coordinates is
\[ \dd s^2 = -\dd t^2 + \frac{\dot{R}^2}{1 + 2E} \dd r^2 + R^2\dd \Omega^2 \]
where $R$ is a function of $r$ and $t$, $E$ is a function of $r$, and
\[ \dd \Omega^2 = \dd \theta^2 + \sin^2(\theta)\dd \phi^2. \]

For this project we used Painlev\'{e}-Gullstrand coordinates instead of Lema\^{i}tre
coordinates. The Lema\^{i}tre-Tolman-Bondi metric in Painlev\'{e}-Gullstrand coordinates
is
\[ \dd s^2 = \dd t^2 + \frac{(E^b)^2}{x^2} (\dd x + N^x\dd t^2)^2 + x^2 \dd \Omega^2 \]
where
\[ N^x = -\frac{x}{\gamma\sqrt{\Delta}} \sin\left(\frac{\sqrt{\Delta}b}{x}\right)
\cos\left(\frac{\sqrt{\Delta}b}{x}\right). \]
Note that here $x$ is the radial coordinate, so $x \ge 0$.

For more details on The Lema\^{i}tre-Tolman-Bondi metric in Painlev\'{e}-Gullstrand
coordinates, see Appendix B and C of the paper ``LTB spacetimes in terms of Dirac
observables'' by K. Giesel et al.\ \cite{metric}.

Then we apply the loop quantum gravity corrections to get the equations of motion
\[ \dot{b} = \frac{\gamma x}{2(E^b)^2} - \frac{\gamma}{2x} - \frac{x}{\gamma\Delta}
\sin\left( \frac{\sqrt{\Delta}}{x}b \right) \left( \frac{3}{2}\sin\left(
\frac{\sqrt{\Delta}}{x}b \right) + \partial_x \left(
\sin\left(\frac{\sqrt{\Delta}}{x}b\right) \right) \right) \]
and
\[ \dot{E^b} = -\frac{x^2}{\gamma\sqrt{\Delta}} \partial_x \left( \frac{E^b}{x} \right)
\sin \left( \frac{\sqrt{\Delta}}{x}b \right) \cos \left( \frac{\sqrt{\Delta}}{x}b
\right). \]
More information about the loop quantum gravity corrections and these equations of
motion can be found in these papers \cite{lqg2, lqg1}.

\section{The Simulation Program}

Everything in section was done during the summer before the start of the school year
while working, but it is central to my project, so I will describe it here in detail.

The simulation is a result of numerically solving the equations of motion. The methods
used to solve these coupled partial differential equations are adopted from the book
\textit{Finite Volume Methods for Hyperbolic Problems} by Randall J. LeVeque \cite{fvm}.
The differential equations need to be rewritten to a form that these methods can more
easily be applied to first though.

\subsection{The PDEs}

The relevant partial differential equations are
\[ \dot{b} = \frac{\gamma x}{2(E^b)^2} - \frac{\gamma}{2x} - \frac{x}{\gamma\Delta}
\sin\left( \frac{\sqrt{\Delta}}{x}b \right) \left( \frac{3}{2}\sin\left(
\frac{\sqrt{\Delta}}{x}b \right) + \partial_x \left(
\sin\left(\frac{\sqrt{\Delta}}{x}b\right) \right) \right) \]
and
\[ \dot{E^b} = -\frac{x^2}{\gamma\sqrt{\Delta}} \partial_x \left( \frac{E^b}{x} \right)
\sin \left( \frac{\sqrt{\Delta}}{x}b \right) \cos \left( \frac{\sqrt{\Delta}}{x}b
\right) \]
where a dot is used for the derivative with respect to time. To simplify things, we can
introduce the variables
\[ \beta = \frac{\sqrt{\Delta}}{x}b \]
and
\[ \alpha = \gamma\sqrt{\Delta} \left( \frac{1}{x^2} - \frac{1}{(E^b)^2} \right). \]
By substituting these variable in for $b$ and $E^b$, we get the equations
\[ \dot{\alpha} = -\frac{x\sin(2\beta)}{2\gamma\sqrt{\Delta}}\alpha_x -
\frac{\sin(2\beta)\alpha}{\gamma\sqrt{\Delta}} \]
and
\[ \dot{\beta} = -\frac{x\sin(2\beta)}{2\gamma\sqrt{\Delta}}\beta_x -
\frac{3\sin^2(\beta)}{2\gamma\sqrt{\Delta}} - \frac{\alpha}{2}. \]
Furthermore, let us set $\Delta = \gamma = 1$ to get
\begin{equation*}
    \dot{\alpha} = -\frac{x}{2}\sin(2\beta)\alpha_x - \sin(2\beta)\alpha
\end{equation*}
and
\begin{equation*}
    \dot{\beta} = -\frac{x}{2}\sin(2\beta)\beta_x - \frac{3}{2}\sin^2(\beta) -
    \frac{\alpha}{2}.
\end{equation*}

\subsection{The PDE Solver}

The basic method used here is Godunov's method, described in section 12.1 of LeVeque's
book, but implemented in terms of fluctuations and wave speeds as in section 12.2 of
LeVeque's book. However, the material in sections 16.1, 16.4, 16.5, 16.6, and 9.4 are
also necessary, along with any background needed for these sections from the book.

The basic idea of this method is that you break your domain up into cells and
approximate the functions as piecewise constant functions. At each time step, you
calculate the fluctuations at each cell interface to see how each cell effects its
neighbor.

Since these methods were developed with fluid dynamics in mind, the cells are often
thought of as having a ``fluid'' with the discontinuities between cells causing waves in
the fluid which leads to the fluctuations. I will adopt these ideas for the explanation
since it makes things easier to grasps and is also what is done in LeVeque's book.

For these equations, the wave speeds are given by the expression
\[ \frac{x}{2}\sin(2\beta) \]
which does not depend on $\alpha$, so that variable can be ignored for the moment.
Unfortunately, this expression is nonlinear and not monotonic, so it leads to
complicated wave structures involving combinations of shock waves and rarefaction waves,
called compound waves. More details about this is found in section 16.1 of LeVeque's
book.

Luckily, we only need to calculate fluctuations for this method, so we do not need to
calculate the exact wave structure. Instead, we can just find the average wave speed to
the right and the average wave speed to the left. If the wave moves in only one
direction, then this is easy, the average wave speed of the $n$\textsuperscript{th} cell
interface is given by
\begin{equation} \label{eq:sn}
    s_n = \begin{dcases}
        \frac{x_n}{2}\frac{\sin^2(\beta_r) - \sin^2(\beta_l)}{\beta_r - \beta_l}, &
        \beta_l \ne \beta_r \\
        \frac{x_n}{2} \sin(2\beta_l), & \beta_l = \beta_r
    \end{dcases}
\end{equation}
where $\beta_l$ and $\beta_r$ are the values of $\beta$ in the cells to the left and
right of the cell interface respectively, and $x_n$ is the location of the
$n$\textsuperscript{th} cell interface. In this case, a positive value is a wave moving
to the left and a negative value is a wave moving to the right. Since everything is
moving in one direction, the wave size is given by $\beta_r - \beta_l$, and the wave
speed times the wave size gives the fluctuations.

Now for waves that have components moving in both directions, things become more
complicated. First, this case happens if and only if the wave speed at $\beta_l$ is
negative, and the wave speed at $\beta_r$ is positive. Since $x$ is positive, the sign
of the wave speed depends only on $\sin(2\beta)$. Now, let us split this case into two
cases. Also, for the following discussion, let an increasing zero be a zero of
$\sin(2\beta)$ where it is increasing, and let a decreasing zero be a zero where it is
decreasing.

In the case where $\beta_l < \beta_r$, the condition is equivalent to there being an
increasing zero in $(\beta_l,\beta_r)$. We also know that for $\sin(2\beta)$ the
increasing zeros are at $k\pi$ for some $k\in\mathbb{Z}$. Thus, a necessary and
sufficient condition for this case is
\[ \left\lfloor \frac{\beta_l}{\pi} \right\rfloor < \left\lfloor \frac{\beta_r}{\pi}
\right\rfloor \]
and the average wave speeds to the left and right are given by
\[ s_n^l = \frac{x_n}{2} \frac{\sin^2(\beta_l^*) - \sin^2(\beta_l)}{\beta_l^* - \beta_l}
\]
and
\[ s_n^r = \frac{x_n}{2} \frac{\sin^2(\beta_r) - \sin^2(\beta_r^*)}{\beta_r - \beta_r^*}
\]
respectively, where $\beta_l^*$ and $\beta_r^*$ are the nearest increasing zeros to
$\beta_l$ and $\beta_r$ respectively. While $\beta_l^*$ and $\beta_r^*$ are not
necessarily equal, if they are not then they are connected by a stationary shock wave
which does not affect the fluctuations.

From the locations of the increasing zeros, we know that $\sin^2(\beta_l^*) =
\sin^2(\beta_r^*) = 0$. The wave size for these waves are $\beta_l^* - \beta_l$ for the
left moving wave and $\beta_r - \beta_r^*$ for the right moving wave. However, since we
only care the fluctuations, we can scale the wave speeds and sizes so that we no longer
need to calculate $\beta_l^*$ or $\beta_r^*$, as long as it leads to the same
fluctuations. Let us use the effective speeds
\[ s_n^l = \frac{x_n}{2} \frac{\sin^2(\beta_l)}{\beta_l - \beta_r} \]
and
\[ s_n^r = \frac{x_n}{2} \frac{\sin^2(\beta_r)}{\beta_r - \beta_l} \]
with effective wave size of $\beta_r - \beta_l$ for both. This has the benefit of being
the same wave size as the case where the wave is moving in only one direction. Note that
this scaling of the wave speeds may not work if higher order corrections are added to
this method.

Now, for the case where $\beta_l > \beta_r$ there needs to be a decreasing zero in
$(\beta_r,\beta_l)$. The decreasing zeros are at $(k + 1/2)\pi$ for some
$k\in\mathbb{Z}$, so the necessary and sufficient condition for this case is
\[ \left\lfloor \frac{\beta_l}{\pi} - \frac{1}{2} \right\rfloor  > \left\lfloor
\frac{\beta_r}{\pi} - \frac{1}{2} \right\rfloor \]
with the average wave speeds given by the same formulas as the previous case, except
here $\beta_l^*$ and $\beta_r^*$ are decreasing zeros and so $\sin^2(\beta_l^*) =
\sin(\beta_r^*) = 1$. After scaling the speeds in the same way as the last case, the
effective wave speeds are
\[ s_n^l = \frac{x_n}{2} \frac{\cos^2(\beta_l)}{\beta_r - \beta_l} \]
and
\[ s_n^r = \frac{x_n}{2} \frac{\cos^2(\beta_r)}{\beta_l - \beta_r} \]
with effective wave size of $\beta_r - \beta_l$ for both again.

Combining all of this together, the simplest way to model this is: at each cell
interface we have a wave of size $\beta_r - \beta_l$ moving to both the left and right.
The speed of the waves at the $n$\textsuperscript{th} cell interface are given by
\[ s_n^l = \begin{dcases}
    \frac{x_n}{2} \frac{\sin^2(\beta_l)}{\beta_l - \beta_r}, & \left\lfloor
    \frac{\beta_l}{\pi} \right\rfloor < \left\lfloor \frac{\beta_r}{\pi} \right\rfloor
    \\
    \frac{x_n}{2} \frac{\cos^2(\beta_l)}{\beta_r - \beta_l}, & \left\lfloor
    \frac{\beta_l}{\pi} - \frac{1}{2} \right\rfloor  > \left\lfloor
    \frac{\beta_r}{\pi} - \frac{1}{2} \right\rfloor \\
    \min(s_n, 0), & \textrm{otherwise}
\end{dcases} \]
and
\[ s_n^r = \begin{dcases}
    \frac{x_n}{2} \frac{\sin^2(\beta_r)}{\beta_r - \beta_l}, & \left\lfloor
    \frac{\beta_l}{\pi} \right\rfloor < \left\lfloor \frac{\beta_r}{\pi} \right\rfloor
    \\
    \frac{x_n}{2} \frac{\cos^2(\beta_r)}{\beta_l - \beta_r}, & \left\lfloor
    \frac{\beta_l}{\pi} - \frac{1}{2} \right\rfloor  > \left\lfloor
    \frac{\beta_r}{\pi} - \frac{1}{2} \right\rfloor \\
    \max(s_n, 0), & \textrm{otherwise}
\end{dcases} \]
where $s_n$ is given by equation \ref{eq:sn}. The wave speeds may be zero which is
equivalent to having no wave in that direction. The fluctuations to the left and right
are given by multiplying these wave speeds by $\beta_r - \beta_l$.

Now we turn to the $\alpha$ variable. It has the same wave speeds as $\beta$, so we can
use the same wave speeds $s_n^l$ and $s_n^r$ that we calculated with $\beta$, but with
wave size of $\alpha_r - \alpha_l$, that is we multiply the wave speeds by that to get
the fluctuations in $\alpha$.

Now that we can calculate fluctuations, we need a formula for updating $\alpha$ and
$\beta$ over a time step $\Delta t$. To distinguish between indices for quantities
related to cells and quantities related to cell interfaces, let cell quantities be
indexed by $n$ and cell interface quantities be indexed $n-1/2$. For example, the value
of $\beta$ in the $n$\textsuperscript{th} cell will be $\beta_n$ while the wave speed to
the right at the cell interface of the $n-1$ and $n$ cells will be $s_{n-1/2}^r$.
Especially important is that the midpoints of each cell will now be labeled $x_n$ while
the locations of the cell interfaces will be labeled $x_{n-1/2}$. So now we have
\[ s_{n-1/2}^l = \begin{dcases}
    \frac{x_{n-1/2}}{2} \frac{\sin^2(\beta_{n-1})}{\beta_{n-1} - \beta_n}, &
    \left\lfloor \frac{\beta_{n-1}}{\pi} \right\rfloor < \left\lfloor
    \frac{\beta_n}{\pi} \right\rfloor \\
    \frac{x_{n-1/2}}{2} \frac{\cos^2(\beta_{n-1})}{\beta_n - \beta_{n-1}}, &
    \left\lfloor \frac{\beta_{n-1}}{\pi} - \frac{1}{2} \right\rfloor  > \left\lfloor
    \frac{\beta_n}{\pi} - \frac{1}{2} \right\rfloor \\
    \min(s_{n-1/2}, 0), & \textrm{otherwise}
\end{dcases} \]
and
\[ s_{n-1/2}^r = \begin{dcases}
    \frac{x_{n-1/2}}{2} \frac{\sin^2(\beta_n)}{\beta_n - \beta_{n-1}}, & \left\lfloor
    \frac{\beta_{n-1}}{\pi} \right\rfloor < \left\lfloor \frac{\beta_n}{\pi}
    \right\rfloor \\
    \frac{x_{n-1/2}}{2} \frac{\cos^2(\beta_n)}{\beta_{n-1} - \beta_n}, & \left\lfloor
    \frac{\beta_{n-1}}{\pi} - \frac{1}{2} \right\rfloor  > \left\lfloor
    \frac{\beta_n}{\pi} - \frac{1}{2} \right\rfloor \\
    \max(s_{n-1/2}, 0), & \textrm{otherwise}
\end{dcases} \]
where
\[ s_{n-1/2} = \begin{dcases}
    \frac{x_{n-1/2}}{2} \frac{\sin^2(\beta_n) - \sin^2(\beta_{n-1})}{\beta_n -
    \beta_{n-1}}, & \beta_{n-1} \ne \beta_n \\
    \frac{x_{n-1/2}}{2} \sin(2\beta_n), & \beta_{n-1} = \beta_n
\end{dcases}. \]

Now the regular update formula for Godunov's method in terms of fluctuations is
\[ \beta_n^{new} = \beta_n - \frac{\Delta t}{\Delta x} (s_{n-1/2}^r (\beta_n -
\beta_{n-1}) + s_{n+1/2} (\beta_{n+1} - \beta_n) ) \]
but we have other terms, known as source terms that need to be accounted for as well. I
used the simplest possible way to account for them, which is to just use the values
$\beta_n$ and $\alpha_n$ in them, but other more sophisticated methods are discussed
in chapter 17 of LeVeque's book. This gives the update formulas
\[ \alpha_n^{new} = \alpha_n - \Delta t \left( \frac{s_{n-1/2}^r (\alpha_n -
\alpha_{n-1}) + s_{n+1/2}^l (\alpha_{n+1} - \alpha_n)}{\Delta x} +
\sin(2\beta_n)\alpha_n \right) \]
and
\[ \beta_n^{new} = \beta_n - \Delta t \left( \frac{s_{n-1/2}^r (\beta_n - \beta_{n-1}) +
s_{n+1/2}^l (\beta_{n+1} - \beta_n)}{\Delta x} + \frac{3}{2}\sin^2(\beta_n) +
\frac{\alpha_n}{2} \right) \]

\subsection{Boundary Conditions}

This is almost a complete description of the solver, the only part left is how to deal
with the boundaries. The best way to apply boundary conditions is to specify how to
calculate $\alpha_0$, $\beta_0$, $\alpha_{N+1}$, and $\beta_{N+1}$ assuming you are
given a collection of $N$ samples of $\beta$ and $\alpha$, that is you know $\alpha_1$
to $\alpha_N$ and $\beta_1$ to $\beta_N$. Then you can just use these values in the same
update formula as all the rest of the cells.

The boundary condition at the origin turns out to not affect the numerics at all for
this method. This is because the wave speeds depend linearly on $x$, so when $x=0$ the
wave speeds are also zero. Thus, no matter the choice of $\alpha_0$ and $\beta_0$ you
will get the same solution since these numbers will be multiplied by zero. However, it
is easier to implement with a boundary condition at the origin anyway, so
$\alpha_0=\alpha_1$ and $\beta_0=\beta_1$ was used.

The outer boundary is much more complicated however, and can effect the numerics. Since
the solution technically could extend to infinity, there is no true boundary condition
to use, we just need to pick a boundary condition that approximates the behavior we
would get if we extended the solution range. We decided to use the conditions that
$\dot{\alpha} = \dot{\beta} = 0$. This would cause the density to be zero at the
boundary as well. This is equivalent to the condition that $\alpha_N^{new} = \alpha_N$
and $\beta_N^{new} = \beta_N$, which turns out to be satisfied by
\[ \beta_{N+1} = -\sin^{-1}\left( \sqrt{\sin^2(\beta_N) - \frac{1}{N}
(3\sin^2(\beta_N) + \alpha_N)} \right) \]
and
\[ \alpha_{N+1} = \frac{1 + 2 (\beta_{N+1} - \beta_N) \sin(2\beta_N)}{3\sin^2(\beta_N) +
\alpha_N} \alpha_N \]
but a check must be included to make sure we do not take the square root of a negative
number as we are looking for a real solution.

\section{Examining the Dynamics}

Now that we have a simulation program we need to test it and see if there is anything
interesting we can learn from it. While there is no analytic solution to these equations
to compare against to test the program, previous work has been done on a special case of
these equations \cite{lqg2, lqg1}. This special case is the case of a flat space-time,
which in these equations corresponds to $\alpha = 0$.

Beyond that, we can also make hypotheses of how the simulation should go based on their
underlying differential equations and test them using the program. If the program agrees
with our hypothesis, that would suggest the program is likely correct, but if they do
not agree, then we are still left wondering if our hypothesis is wrong or if the program
is wrong.

\subsection{Special Case}

There were two main things we were looking for in the dynamics of the simulation in the
special case: the solution stays flat, and the solution start with a phase where the
matter collapses inwards, followed by a phase where the matter expands outwards lead by
a shock wave.

To test we used initial conditions that $\alpha = 0$ and the initial density is given by
\[ \rho(x) = \begin{dcases}
    \rho_0, & x < x_0 \\
    0, & x > x_0
\end{dcases} \]
for some $x_0$. In general, the density is related to $\alpha$ and $\beta$ by
\[ \rho = \frac{1}{8\pi x^2} \frac{\partial}{\partial x} (x^3 (\sin^2(\beta) + \alpha))
\]
so
\[ x^3(\sin^2(\beta) + \alpha) = 8\pi \int_0^x\! s^2 \rho(s) \dd s. \]
In this case,
\[ x^3\sin^2(\beta) = \begin{dcases}
    \frac{8}{3}\pi\rho_0 x^3, & x < x_0 \\
    \frac{8}{3}\pi\rho_0 x_0^3, & x > x_0
\end{dcases} \]
thus
\[ \beta = \begin{dcases}
    -\sin^{-1}\left(\sqrt{\frac{8}{3}\pi\rho_0}\right), & x < x_0 \\
    -\sin^{-1}\left(\sqrt{\frac{8}{3}\pi\rho_0 \left(\frac{x_0}{x}\right)^3}\right), & x
    > x_0
\end{dcases} \]
for the initial condition of $\beta$.

After writing a script that applied the solver to these initial conditions and played an
animation of the results, we found that the solver did behave according to our
expectations in this case. We also tried different distributions of $\rho$, such as a
Gaussian, and they all behaved in this manner.

\subsection{Simple Curvature}

The simple curvature case we decided to test used the same initial density as the
special case, but with the condition that
\[ E^b = \begin{dcases}
    \frac{xa_0}{\sqrt{a_0^2 - x^2}}, & x < x_0 \\
    \frac{xa_0}{\sqrt{a_0^2 - x_0^2}}, & x > x_0
\end{dcases} \]
for some parameter $a>x_0$. This initial condition is curved inside $x_0$ and flat
outside it. In terms of $\alpha$ this corresponds to
\[ \alpha = \begin{dcases}
    \frac{1}{a_0^2}, & x < x_0 \\
    \left( \frac{x_0}{a_0x} \right)^2, & x > x_0
\end{dcases} \]

Our expectations for this case involved the same collapse and expansion phases from the
special case, but with a re-collapse at some point. We also expected the solution to
stay curved where the was matter and remain flat where there is no matter.

After writing another script to run the solver on these initial conditions and animate
the results, we found that there was the same collapse and expansion as we expected, but
no re-collapse could be found no matter how long we ran the solver for. On the other
hand, the solution did indeed stay curved and flat where it was supposed to.

This was the first case where we had to figure out if it was our expectations that were
incorrect, or if there was an error in the program. After reexamining the differential
equations, we decided that in this case it was likely our expectations that were wrong.

We also tried the same case except with the sign of $\alpha$ flipped and found that it
acted essentially the same as the other sign, which is what we expected.

\subsection{More Complicated Curved Cases}

Having tried curved cases with $\alpha$ either all positive or all negative, we wanted
to test more complicated curved cases with $\alpha$ both positive and negative. The
first test we did was $\alpha$ as a sinusoidal. Specifically, we used
\[ \alpha = \begin{dcases}
    \frac{2\rho_0x_0}{3x} \sin\left(\frac{4\pi x_0}{x}\right), & x < x_0 \\
    0, & x > x_0
\end{dcases} \]
and the same initial $\rho$. We did not have many expectations for this case, but note
that in all the previous cases we tested $\beta > -\pi$ had always remained true, and we
expected this to be the case here as well.

After writing the test script, the solution acted mostly how we thought it might, but
with the notable exception that $\beta$ dropped below $-\pi$ near the origin. To test if
this was just an issue with the program, we ran the same test with several grid sizes
and counted the number of data points that dropped below $-\pi$. If the number of data
points below $-\pi$ was constant while varying the grid size, then this would indicate
it is just a flaw in the program, but unfortunately this was not the case and number of
points below $-\pi$ scaled with grid size.

While this did not prove that the program worked and that this phenomenon as actually
part of the dynamics, it certainly made us question it. Upon further inspection of the
differential equations, if $\beta\to-\pi$ and $\alpha\to0$ then $\beta$ should not drop
below $-\pi$, and this is the situation in the previous simpler case, but if $\alpha$
does not approach $0$, then it is much more complicated as in the limit $\dot{\beta} =
\frac{\alpha}{2}$.

To get a better understanding of the phenomenon, and to hopefully find out if it is an
error or not, we decided to test more cases and see if it occurs again. The next case we
tested was $\alpha$ in the form of a sawtooth curve, specifically
\[ \alpha = \begin{dcases}
    \frac{x}{4}, & x < \frac{x_0}{4} \\
    \frac{x_0}{8} - \frac{x}{4}, & \frac{x_0}{4} < x < \frac{3x_0}{4} \\
    \frac{x}{4} - \frac{x_0}{4}, & \frac{3x_0}{4} < x < x_0 \\
    0, & x_0 < x
\end{dcases} \]
and again $\beta$ dropped below $-\pi$. Although, this time it was not near the origin
where it dropped below, and there was a shock wave travelling towards the origin from
where it dropped below.

Wanting further initial conditions to examine, we decided to create a script that
generates a random initial $\alpha$, uses it in the solver, and plays an animation of
the result. It created the random $\alpha$ by generating uniformly distributed random
data across the valid interval for $\alpha$ and then apply a Gaussian smoothing filter
to smooth out the data.

It was at this point we ran into a problem, since the boundary condition used has a
square root in it, it is possible to have complex numbers in our solution while we want
a real solution. In practice, we simply stop the solver whenever the square root would
give an imaginary number, and this was happening nearly every time with the random
initial conditions. This meant that we could not get to the part of the dynamics we were
interested in.

We tried a few easy fixes to see if we could just work around this, but nothing we tried
really helped. We would have needed to completely change and reimplement the boundary
condition, but instead we shifted our focus to other things, and this is where the
program still stands. That said, we believe that this boundary condition issue did not
have a significant effect on the dynamics, it just got in the way of solving further
times occasionally.

\section{Event Horizons Lifetimes}

Our next goal was to examine the lifetimes of the event horizons of the black holes and
how they related to the mass of the black hole. To this end, I created a function that
when given initial conditions will time the lifetime of the event horizon. We used this
function and the initial conditions from the special case to collect data of event
horizon lifetime and mass by varying the $\rho_0$ parameter in the initial conditions to
vary the mass.

This case has already been studied, so it is known that the lifetime depends
quadratically on the mass, and the leading coefficient is $\frac{8}{3}\pi = 8.37758$
\cite{lqg2}. This provides a good first test to make sure everything is working.
Unfortunately, when this test is run using this method, a leading coefficient of
$2.3703$ is obtained.

We believe we know the reason for the discrepancy though. Since the solutions to these
differential equations involve shock waves, they have discontinuities, so the equations
need to be converted to integral equations to cope with that. In LeVeque's book this is
facilitated by treating the equations as conservation laws, and using that to calculate
the shock speed.

This means that the methods can give different results depending on the form of the
equations you apply them to, even if those forms are mathematically equivalent, this
picking the correct variables to use in thee solve is crucial. The variables I used were
chosen because they are constant in the simple curved case, and these methods are
designed to keep constant section of the solution constant. I also treated the equation
as non-conservative, since that was easier to implement.

This decision had its merits, but it also caused the speed of shock waves to be faster
than they should have been, which is what we believe caused the event horizon lifetimes
to differ. We believe that most of what we observed qualitatively in the dynamics is
still correct, it is just that everything is sped up somewhat.

\section{Next Steps}

The first step that would need to be done next is to rewrite the differential equations
and program to fix the shock wave speed issue. We think it can be fixed by defining the
variable
\[ B = xb \]
and using the equations
\[ \dot{\alpha} = -\frac{x}{2}\sin\left(\frac{2B}{x^2}\right)\alpha_x -
\sin\left(\frac{2B}{x^2}\right)\alpha \]
and
\[ \dot{B} = -\frac{\partial}{\partial x} \left( \frac{x^3}{2}
\sin\left(\frac{B}{x^2}\right) \right) + \frac{\alpha}{2} \]
This is a proper conservation law in $B$, albeit with a source term, and $\alpha$ can
once again be treated using the material from section 16.6 of LeVeque's book, similar to
how it is treated in the previous method. It may also be beneficial to change from
$\alpha$ to a different variable as well, but this one seems good. The variable used
should be one that is constant when some desired property is held since using this
method will cause constant sections to remain constant. See section 16.6 of LeVeque's
book for an example of choosing a desirable variable.

The next thing to do would be to find a better boundary condition that does not break
down like the current one does. My idea for how to do this was to calculate the solution
to a very large radius so that the boundary condition does not have to propagate into
the desired solution domain. Then you could truncate the solution down to the desired
domain and calculate the solution with a variety of boundary conditions to find one that
acts similar to the solution where effects from the boundary condition are removed.

From there it should be confirmed that the new method gives the correct event horizon
lifetimes. After that, you could look at event horizon lifetimes as function of mass for
the simple curved case. You could also look into the $\beta$ variable dropping below
$-\pi$ more, assuming that still happens after the change of variables.

Another avenue that could be pursued is to try to improve the accuracy of the method by
using improved methods. Specifically, it might be a good idea to try a more
sophisticated method to deal with the source terms like a fractional step method, and to
see if that changes the dynamics any. Similarly, you could add higher order correction
terms to the methods and see if that changes anything significantly.

\section{Conclusion}

By using the Lema\^{i}tre-Tolman-Bondi metric in Painlev\'{e}-Gullstrand coordinates and
some loop quantum gravity corrections, we got a set of coupled partial differential
equations that describe a black hole as a dust cloud. By creating a solver for these
equations, we effectively created a program that simulated black holes as dust clouds.

The solver builds off ideas from Godunov's method, but uses wave speeds and fluctuations
to make coupling the equations together easier. The methods used are very dependent on
the equations used, and especially the variables used. Unfortunately, it turns out the
choice of variables made early on was not the best choice of variables, but they still
revealed some interesting dynamics.

The simulation that were run on certain cases revealed that after the initial collapse
and expansion, there may not be a re-collapse afterwords, as was once thought. It also
revealed that the $\beta$ variable used may drop below $-\pi$ which was surprising.
However, this is even less conclusive than the first observation as it was not fully
investigated due to time constraints.

With a few fixes that would not be too difficult to implement, the solver could be
improved quite a bit. This would allow a much better picture of the dynamics, both
qualitatively and quantitatively, to be realized. There are many interesting paths this
project could be taken with some more time, and hopefully some very interesting
conclusions just waiting to be found.

\printbibliography[heading=bibintoc]

\end{document}
